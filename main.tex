\documentclass[12pt]{article}
\usepackage[utf8]{inputenc}
\usepackage[finnish]{babel}
\usepackage[T1]{fontenc}
\usepackage[utf8]{inputenc}
\setlength{\parskip}{\medskipamount}
\setlength{\parindent}{0pt}
\setlength{\emergencystretch}{15pt}
\thispagestyle{empty}                   

\newcommand{\ee}[1]{e^{#1}}             % makromäärittely;
                                        % hakasuluissa parametrien lukumäärä

\newcommand{\viiva}{\mathop{\Bigg/}}
\newcommand{\sij}[3]{\viiva\limits_{\hspace*{-5mm}{#1}}^{\hspace*{5mm}{#2}}{#3}}

\title{Gamma funktio}
\author{Reetta Tervo}
\begin{document}
\maketitle

\newpage
%\setcounter{tocdepth}{1}
\tableofcontents

\newpage
\section{Johdanto}
Tämän tutkielman vatoitteena on esitellä gammafunktio, siihen liittyviä määritelmiä ja joitakin sovellustarkoituksia. Keskeisenä osana gammafunktiota on Bohr-Mollerupin lause, joka todistetaan...
\newline
Gammafunktio on ??eräänlainen?? kertoma, joka on määritelty kaikilla muilla luvuilla paitsi positiivisilla kokonaisluvuilla. 
\newpage
\section{Bohr-Mollerupin lause}
Bohr-Mollerupin lauseen mukaan gamma funktio on ainoa kertomafunktion yleistys, joka on logaritmisesti konveksi. (WIKIPEDIA???)


\newpage
\section{Eulerin integraali}
Yleinen esitysmuoto gamma funktiolle on Leonard Eulerin määrittämään Eulerin integraaliin. Tällöin gamma funktio esitetään seuraavasti \newline
$\gamma(x) = \int_{0}^{\infty} e^{-t} t^{x-1} dt$.
\newpage
\section{Gamma funktio}\label{luk: gammafunktio}
Gamma funktioon päästään derivoimalla toistuvasti yhälöä
\newline $\int_{0}^{\infty} e^{-xt} dt$ = $\frac{1}{x}$.
\newpage
\section{Sovelluskohteita????}
\newpage
\section{Kirjallisuutta}


\end{document}