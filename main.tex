\documentclass[12pt]{article}
\usepackage[utf8]{inputenc}
\usepackage[finnish]{babel}
\usepackage[T1]{fontenc}
\usepackage[utf8]{inputenc}
\usepackage{setspace}
\usepackage{amsmath}
\usepackage{amssymb}
\setlength{\parskip}{\medskipamount}
\setlength{\parindent}{0pt}
\setlength{\emergencystretch}{15pt}
\thispagestyle{empty}                   

\newcommand{\ee}[1]{e^{#1}}             % makromäärittely;
                                        % hakasuluissa parametrien lukumäärä

\newcommand{\viiva}{\mathop{\Bigg/}}
\newcommand{\sij}[3]{\viiva\limits_{\hspace*{-5mm}{#1}}^{\hspace*{5mm}{#2}}{#3}}

\title{Gamma funktio}
\author{Reetta Tervo}
\begin{document}
\maketitle

\newpage
%\setcounter{tocdepth}{1}
\tableofcontents

\newpage
\section{Johdanto}
\onehalfspacing
Tämän tutkielman vatoitteena on esitellä gammafunktio, siihen liittyviä määritelmiä ja joitakin sovellustarkoituksia. Keskeisenä osana gammafunktiota on Bohr-Mollerupin lause, joka todistetaan...
\newline
Gammafunktio on ??eräänlainen?? kertoma, joka on määritelty kaikilla muilla luvuilla paitsi positiivisilla kokonaisluvuilla. 
\newpage
\section{Bohr-Mollerupin lause}
Bohr-Mollerupin lauseen mukaan gammafunktio on ainoa kertomafunktion yleistys, joka on logaritmisesti konveksi. 
\newpage


\section{Eulerin integraali}
Yleisin esitysmuoto gammafunktiolle on Leonard Eulerin määrittämä integraali, joka tunnetaan paremmin nimellä Eulerin toinen integraali. Tässä muodossa esitettynä gammafunktio näyttää seuraavalta:
\begin{align*}
    \Gamma(z) = \int_{0}^{\infty} e^{-t} t^{z-1} dt.
\end{align*}
Tämä Eulerin integraali on määritelty kaikilla muilla luvuilla paitsi ei-positiivisilla kokonaisluvuilla. Integraali on siis määritelty myös kompleksi luvuilla, joiden reaaliosan tulee kuitenkin olla positiivinen. \newline
Tutkitaan integraalia arvoilla $z=x+1$ ja $z=x$. \newline
Osittaisintegroidaan integraali
\begin{align*}
    \Gamma(x+1) & = \int_{0}^{\infty} e^{-t} t^{x} dt \\
    & = 
\end{align*}
Nyt osittais integroidaan integraali \newline
\begin{align*}
    \Gamma(x) & = \int_{0}^{\infty} e^{-t} t^{x-1} dt \\
& =
\end{align*}
Huomataan, että $\Gamma(x+1) = x\Gamma(x)$ ja $\Gamma(1) = 1.$ \newline
\newpage

\section{Gamma funktio}\label{luk: gammafunktio}
\onehalfspacing
Gamma funktioon päästään derivoimalla toistuvasti yhälöä \newline
$\int_{0}^{\infty} e^{-xt} dt$ = $\frac{1}{x}$.
\newline
Gamma funktio on erinomainen työkalu laskiessa jonkin ei-positiivisen luvun kertomaa. 
\newpage
\section{Sovelluskohteita????}
\newpage
\section{Kirjallisuutta}


\end{document}