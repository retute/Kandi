\documentclass[12pt]{article}
\usepackage[utf8]{inputenc}
\usepackage[finnish]{babel}
\usepackage[T1]{fontenc}
\usepackage[utf8]{inputenc}
\usepackage{setspace}
\usepackage{amsmath}
\usepackage{amssymb}
\usepackage{amsthm}
\theoremstyle{definition}
\newtheorem{maar}{Määritelmä}
\theoremstyle{plain}
\newtheorem{lause}[maar]{Lause}
\usepackage{csquotes}
\setlength{\parskip}{\medskipamount}
\setlength{\parindent}{0pt}
\setlength{\emergencystretch}{15pt}
\thispagestyle{empty}                   

\newcommand{\ee}[1]{e^{#1}}             % makromäärittely;
                                        % hakasuluissa parametrien lukumäärä

\newcommand{\viiva}{\mathop{\Bigg/}}
\newcommand{\sij}[3]{\viiva\limits_{\hspace*{-5mm}{#1}}^{\hspace*{5mm}{#2}}{#3}}

\title{Gamma funktio}
\author{Reetta Tervo}
\begin{document}
\maketitle

\newpage
%\setcounter{tocdepth}{1}
\tableofcontents

\newpage
\section{Johdanto}
\onehalfspacing
Tämän tutkielman vatoitteena on esitellä gammafunktio. Keskeisenä osana gammafunktiota on Bohr-Mollerupin lause, joka todistetaan...

\newpage





\section{Gammafunktio}
Yleisin esitysmuoto gammafunktiolle on Leonard Eulerin vuonna 1729 määrittämä integraaliesitys
\begin{align}
     \Gamma(x) = \int_{0}^{\infty} e^{-t} t^{x-1} dt.
\end{align}

Gammafunktio on epäoleellinen integraali. Tästä johtuen, tutkittaessa sen olemassa oloa, tulee sen olla määritelty hyvin. Tällöin tutkitaan siis määrättyjä integraaleja (a) $\int_{\epsilon}^{1} e^{-t} t^{x-1} dt$ ja (b) $\int_{1}^{\delta} e^{-t} t^{x-1} dt$.\newline

Tutkitaan ensin määrättyä integraalia (a), kun $x>0$:\newline
Kun luku $t$ on positiivinen, niin integtoitava osuus on pienempi kuin $t^{x-1}$. Tällöin voidaan arvioida integraalia ylöspäin seuraavasti:
\begin{align*}
    e^{-t} t^{x-1} < t^{x-1}.
\end{align*} 
Tästä saadaan 
\begin{align*}
    \int_{\epsilon}^{1} e^{-t} t^{x-1} dt < \int_{\epsilon}^{1} t^{x-1} d = \frac{t^x}{x} = \frac{1}{x} - \frac{\epsilon^x}{x}.
\end{align*}
Kun $\epsilon \rightarrow 0$, niin $\frac{\epsilon^x}{x} \rightarrow 0$ ja tällöin myös $\frac{1}{x}-\frac{\epsilon^x}{x} \rightarrow \frac{1}{x}$. Integraali on siis ylhäältä rajoitettu välillä $(0, 1]$. Täten voidaan todeta, että integraali $\int_{\epsilon}^{1} e^{-t} t^{x-1} dt$ suppenee. \newline

Tutkitaan seuraavaksi määrättyä integraalia (b), kun $x>0$: \newline


Gammafunktio on kertomafunktion yleistys, joten sille pätevät täysin samat ehdot kuin kertomafunktiolle. 

\begin{maar}
(kertomafunktio)
Funktio $f: N \rightarrow N$ on kertomafunktio, jos seuraavat ehdot toteutuvat:


\begin{quote}
    (1) $f(1)=1$ \newline
    (2) $f(n+1)=(n+1)f(n)$ \newline
    Kun ehdot toteutuvat, niin voidaan merkitä $f(n)=n!$.
\end{quote}
\end{maar}

\begin{proof}
Tutkitaan integraalia (1) arvoilla $n=x+1$ ja $n=x$. \newline
Osittaisintegroidaan integraali, kun $a\rightarrow0$ ja $b\rightarrow\infty$
\begin{align*}
    \Gamma(x+1) & = \int_{a}^{b} e^{-t} t^{x} dt \\
    & = -e^{-t}t^{x}\Big|_a^b+\int_a^b e^{-t}t^{x-1}dt \\
    & = e^{-a}a^{x}-e^{-b}b^{x}+x\int_a^be^{-t}t^{x-1}dt.
\end{align*}
Kun $a\rightarrow0$, niin $a^x\rightarrow0$ ja tällöin myös $e^{-a}a^x \rightarrow0$. Koska tiedetään,   \newline
Nyt osittais integroidaan integraali \newline
\begin{align*}
    \Gamma(x) & = \int_{0}^{\infty} e^{-t} t^{x-1} dt \\
& =
\end{align*}
Huomataan, että $\Gamma(x+1) = x\Gamma(x)$ ja $\Gamma(1) = 1.$
\end{proof}

\newpage




\section{Bohr-Mollerupin lause}
\onehalfspacing
Bohr-Mollerupin lauseen mukaan gammafunktio on ainoa kertomafunktion yleistys, joka on logaritmisesti konveksi. Tämä ominaisuus tekee gammafunktiosta erityisen.
\newline

\begin{maar}
(konveksisuus)
Olkoon $X \subset \mathbb{R}, t \in (0, 1)$ ja funktio $f: X \rightarrow \mathbb{R}$. Funktio f on \emph{konveksi}, jos kaikilla $x,y \in X$
\begin{align*}
    f(tx+(1-t)y) \le tf(x)+(1-t)f(y).
\end{align*}
\end{maar}


\begin{maar}
(logaritminen konveksisuus)
Olkoon $X \subset \mathbb{R}, t \in (0, 1)$. Funktio $f: X \rightarrow \mathbb{R}$ on sis määritelty ja positiivinen. Funktio f on \emph{logaritmisesti konveksi}, jos $logf(x)$ on \emph{konveksi}.
\end{maar}


\begin{maar}
(gammafunktio)
Koska gammafunktio on kertomafunktion yleistys, toteuttaa se seuraavat ehdot:
\begin{quote}
    (1) $\Gamma(1)=1$ \newline
    (2) $\Gamma(x+1)=x+\Gamma(x)$ \newline
    (3) $\Gamma(x)$ on logaritmisesti konveksi.
\end{quote}
\end{maar}

\begin{lause}
Lause(Born-Mollerupin lause)
\newline
Jos funktio $f(x)$ toteuttaa seuraavat kolme ehtoa, niin se on gammafunktio:
\begin{quote}
(1) $f(1)=1$ \newline
(2) $f(x+1)=xf(x)$ \newline
(3) Funktio $f(x)$ on logaritmisesti konveksi funktio.
\end{quote}
\end{lause}

\begin{proof}

\end{proof}
\newline
Ehdot (1) ja (2) on jo todistettu, joten jäljelle jää todistaa ehto (3) todeksi. 

\newpage
\section{Kirjallisuutta}
\onehalfspacing
[1] 
\newline
[2]
\newline
[3]
\newline
[4]


\end{document}