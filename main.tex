\documentclass[12pt]{article}
\usepackage[utf8]{inputenc}
\usepackage[finnish]{babel}
\usepackage[T1]{fontenc}
\usepackage[utf8]{inputenc}
\usepackage{setspace}
\usepackage{amsmath}
\usepackage{amssymb}
\usepackage{amsthm}
\theoremstyle{definition}
\newtheorem{maar}{Määritelmä}
\theoremstyle{plain}
\newtheorem{lause}[maar]{Lause}
\usepackage{csquotes}
\setlength{\parskip}{\medskipamount}
\setlength{\parindent}{0pt}
\setlength{\emergencystretch}{15pt}
\thispagestyle{empty}                   

\newcommand{\ee}[1]{e^{#1}}             % makromäärittely;
                                        % hakasuluissa parametrien lukumäärä

\newcommand{\viiva}{\mathop{\Bigg/}}
\newcommand{\sij}[3]{\viiva\limits_{\hspace*{-5mm}{#1}}^{\hspace*{5mm}{#2}}{#3}}

\title{Gamma funktio}
\author{Reetta Tervo}
\begin{document}
\maketitle

\newpage
%\setcounter{tocdepth}{1}
\tableofcontents

\newpage
\section{Johdanto}
\onehalfspacing
Tämän tutkielman vatoitteena on esitellä gammafunktio. Keskeisenä osana gammafunktiota on Bohr-Mollerupin lause, joka todistetaan...

\newpage





\section{Gammafunktio}
Yleisin esitysmuoto gammafunktiolle on Leonard Eulerin vuonna 1729 määrittämä integraaliesitys
\begin{equation}\label{yhtalo:gammafunktio}
     \Gamma(x) = \int_{0}^{\infty} e^{-t} t^{x-1} dt.
\end{equation}

Gammafunktio on epäoleellinen integraali. Tästä johtuen, tutkittaessa sen olemassa oloa, tulee sen olla määritelty hyvin. Tällöin tutkitaan siis määrättyjä integraaleja (a) $\int_{\epsilon}^{1} e^{-t} t^{x-1} dt$ ja (b) $\int_{1}^{\delta} e^{-t} t^{x-1} dt$.\newline

Tutkitaan ensin määrättyä integraalia (a), kun $x>0$:\newline
Kun luku $t$ on positiivinen, niin integtoitava osuus on pienempi kuin $t^{x-1}$. Tällöin voidaan arvioida integraalia ylöspäin seuraavasti:
\begin{align*}
    e^{-t} t^{x-1} < t^{x-1}.
\end{align*} 
Tästä saadaan 
\begin{align*}
    \int_{\epsilon}^{1} e^{-t} t^{x-1} dt < \int_{\epsilon}^{1} t^{x-1} d = \frac{t^x}{x} = \frac{1}{x} - \frac{\epsilon^x}{x}.
\end{align*}
Kun $\epsilon \rightarrow 0$, niin $\frac{\epsilon^x}{x} \rightarrow 0$ ja tällöin myös $\frac{1}{x}-\frac{\epsilon^x}{x} \rightarrow \frac{1}{x}$. Kun luku $\epsilon$ lähestyy nollaa, niin integraalin arvo karvaa monotonisesti. Integraali on siis ylhäältä rajoitettu välillä $(0, 1]$ ja raja-arvo
\begin{align*}
    \lim_{\epsilon\to\0}\int_\epsilon^{1}e^{-t}t^{x-1}dt
\end{align*}on olemassa. Täten voidaan todeta, että integraali $\int_{\epsilon}^{1} e^{-t} t^{x-1} dt$ suppenee. \newline

Tutkitaan seuraavaksi määrättyä integraalia (b), kun $x>0$: \newline
Kun luku $t$ on positiivinen, niin kaikki $e^t\ge0$. Tästä seuraa, että kaikilla kokonaisluvuilla $n$ pitää paikkaansa, että $e^t\ge\frac{t^n}{n!}$. Tällöin myös $e^{-t}\le\frac{n!}{t^n}$ ja edelleen $e^{-t}t^{x-1}\le\frac{n!}{t^{n+1-x}}$. Valitaan jonkin $n>x+1$, jolloin saadaan integraalille yläraja $\delta=\frac{n!}{n-x}$ eli
\begin{align*}
    \int_1^{\delta}e^{-t}t^{x-1}dt.
\end{align*}
Nyt integraalin arvo kasvaa, kun $\delta$ kasvaa. Siis raja-arvo $\lim_{1\to\infty}\int_1^{\delta}e^{-t}t^{x-1}dt$ on olemassa.\newline

Koska gammafunktio on kertomafunktion yleistys, niin sille pätevät täysin samat ehdot kuin kertomafunktiolle. 

\begin{maar}
(kertomafunktio)
Funktio $f: N \rightarrow N$ on kertomafunktio, jos seuraavat ehdot toteutuvat:
\begin{quote}
    i) $f(1)=1$ \newline
    ii) $f(n+1)=(n+1)f(n)$ \newline
    Kun ehdot toteutuvat, niin voidaan merkitä $f(n)=n!$.
\end{quote}
\end{maar}

\begin{proof}
Aloitetaan tutkimalla kohtaa i). Integroidaan gammafunktio $f(n)=\Gamma(x)$ arvolla $x=1$.
\begin{equation}
    \Gamma(1) & = \int_\epsilon^\delta e^{-t}t^{1-1} dt \\
    & = \int_\epsilon^\delta e^{-t}dt \\
    & = \Big/_\epsilon^\delta -e^{-t}
\end{equation}
Kun $\epsilon\rightarrow0$ ja $\delta\rightarrow\infty$, niin $-e^{-\delta}\rightarrow0$ ja $-e^{-\epsilon}\rightarrow1$. Siis $\Gamma(1)=1$.

Nyt tutkitaan kohtaa ii) arvoilla $n=x+1$ ja $n=x$, kun $f(n)=\Gamma(n)$. \eqref{yhtalo:gammafunktio} \newline
Osittaisintegroidaan integraali \eqref{yhtalo:gammafunktio}, kun $n=x+1$:
\begin{align*}
    \Gamma(x+1) & = \int_{\epsilon}^{\delta} e^{-t} t^{x} dt \\
    & = \Big/_\epsilon^\delta -e^{-t}t^{x}+\int_\epsilon^\delta e^{-t}t^{x-1}dt \\
    & = e^{-\epsilon}\epsilon^{x}-e^{-\delta}\delta^{x}+x\int_\epsilon^\delta e^{-t}t^{x-1}dt.
\end{align*}
Kun $\epsilon\rightarrow0$, niin $\epsilon^x\rightarrow0$ ja tällöin myös $e^{-\epsilon}\epsilon^x \rightarrow0$. Koska tiedetään, että $\frac{\delta^x}{e^\delta} < \frac{n!}{\delta^{n-x}}$. Nyt, kun $\delta\rightarrow\infty$, niin luku $\frac{n!}{\delta^{n-x}}\rightarrow0.$ Tällöin myös $\frac{\delta^x}{e^\delta}\rightarrow0$ ja toinenkin termi katoaa. Nyt jäljelle jää enää vain \begin{align*}
   \Gamma(x+1) & =  x\int_\epsilon^\delta e^{-t}t^{x-1}dt \\
    & = x\Gamma(x)
\end{align*}

Siis $\Gamma(x+1) = x\Gamma(x)$ ja $\Gamma(1) = 1.$
\end{proof}

\newpage




\section{Bohr-Mollerupin lause}
\onehalfspacing
Bohr-Mollerupin lauseen mukaan gammafunktio on ainoa kertomafunktion yleistys, joka on logaritmisesti konveksi. Tämä ominaisuus tekee gammafunktiosta erityisen.
\newline

\begin{maar}
(konveksisuus)
Olkoon $X \subset \mathbb{R}, t \in (0, 1)$ ja funktio $f: X \rightarrow \mathbb{R}$. Funktio f on \emph{konveksi}, jos kaikilla $x,y \in X$
\begin{align*}
    f(tx+(1-t)y) \le tf(x)+(1-t)f(y).
\end{align*}
\end{maar}


\begin{maar}
(logaritminen konveksisuus)
Olkoon $X \subset \mathbb{R}, t \in (0, 1)$. Funktio $f: X \rightarrow \mathbb{R}$ on siis määritelty ja positiivinen. Funktio f on \emph{logaritmisesti konveksi}, jos $logf(x)$ on \emph{konveksi}.
\end{maar}


\begin{maar}
(gammafunktio)
Koska gammafunktio on kertomafunktion yleistys, toteuttaa se seuraavat ehdot:
\begin{quote}
    (1) $\Gamma(1)=1$ \newline
    (2) $\Gamma(x+1)=x+\Gamma(x)$ \newline
    (3) $\Gamma(x)$ on logaritmisesti konveksi.
\end{quote}
\end{maar}

\begin{lause}
Lause(Bohr-Mollerupin lause)
\newline
Jos funktio $f(x)$ toteuttaa seuraavat kolme ehtoa, niin se on gammafunktio:
\begin{quote}
(1) $f(1)=1$ \newline
(2) $f(x+1)=xf(x)$ \newline
(3) Funktio $f(x)$ on logaritmisesti konveksi funktio.
\end{quote}
\end{lause}

\begin{proof}

\end{proof}
\newline
Ehdot (1) ja (2) on jo todistettu, joten jäljelle jää osoittaa ehto (3) todeksi. 

\newpage
\section{Kirjallisuutta}
\onehalfspacing
[1] 
\newline
[2]
\newline
[3]
\newline
[4]


\end{document}