\documentclass[12pt]{article}
\usepackage[utf8]{inputenc}
\usepackage[finnish]{babel}
\usepackage[T1]{fontenc}
\usepackage[utf8]{inputenc}
\usepackage{setspace}
\usepackage{amsmath}
\usepackage{amssymb}
\usepackage{csquotes}
\setlength{\parskip}{\medskipamount}
\setlength{\parindent}{0pt}
\setlength{\emergencystretch}{15pt}
\thispagestyle{empty}                   

\newcommand{\ee}[1]{e^{#1}}             % makromäärittely;
                                        % hakasuluissa parametrien lukumäärä

\newcommand{\viiva}{\mathop{\Bigg/}}
\newcommand{\sij}[3]{\viiva\limits_{\hspace*{-5mm}{#1}}^{\hspace*{5mm}{#2}}{#3}}

\title{Gamma funktio}
\author{Reetta Tervo}
\begin{document}
\maketitle

\newpage
%\setcounter{tocdepth}{1}
\tableofcontents

\newpage
\section{Johdanto}
\onehalfspacing
Tämän tutkielman vatoitteena on esitellä gammafunktio, siihen liittyviä määritelmiä ja joitakin sovellustarkoituksia. Keskeisenä osana gammafunktiota on Bohr-Mollerupin lause, joka todistetaan...

Gamma funktio on erinomainen työkalu laskiessa jonkin ei-positiivisen luvun kertomaa. 
\newline
Gammafunktio on on kertomafunktion yleistys, joka on määritelty kaikilla muilla luvuilla paitsi ei-positiivisilla kokonaisluvuilla. 
\newpage


\section{Gammafunktio}
Yleisin esitysmuoto gammafunktiolle on Leonard Eulerin vuonna 1729 määrittämä integraaliesitys
\begin{align*}
    \Gamma(z) = \int_{0}^{\infty} e^{-t} t^{z-1} dt.
\end{align*}

Gammafunktio on kertomafunktion yleistys, joten sille pätevät täysin samat ehdot kuin kertomafunktiolle. 

Määritelmä(kertomafunktio) \newline
Funktio $f: N \rightarrow N$ on kertomafunktio, jos seuraavat ehdot toteutuvat:

\begin{quote}
    (1) $f(1)=1$ \newline
    (2) $f(n+1)=(n+1)f(n)$ \newline
    Kun ehdot toteutuvat, niin voidaan merkitä $f(n)=n!$.
\end{quote}
Todistus \newline

Koska gammafunktio on kertomafunktion yleistys, toteuttaa se seuraavat ehdot:

\begin{quote}
    (1) $\Gamma(1)=1$ \newline
    (2) $\Gamma(x+1)=x+\Gamma(x)$ \newline
    (3) $\Gamma(x)$ on logaritmisesti konveksi.
\end{quote}
Tutkitaan integraalia arvoilla $z=x+1$ ja $z=x$. \newline
Osittaisintegroidaan integraali
\begin{align*}
    \Gamma(x+1) & = \int_{0}^{\infty} e^{-t} t^{x} dt \\
    & = 
\end{align*}
Nyt osittais integroidaan integraali \newline
\begin{align*}
    \Gamma(x) & = \int_{0}^{\infty} e^{-t} t^{x-1} dt \\
& =
\end{align*}
Huomataan, että $\Gamma(x+1) = x\Gamma(x)$ ja $\Gamma(1) = 1.$ \newline
\newpage


\section{Bohr-Mollerupin lause}
\onehalfspacing
Bohr-Mollerupin lauseen mukaan gammafunktio on ainoa kertomafunktion yleistys, joka on logaritmisesti konveksi.
\newline

Määritelmä (konveksisuus)
\newline

Lause(Born-Mollerupin lause)
\newline

Jos funktio $f(x)$ toteuttaa seuraavat kolme ehtoa, niin se on gammafunktio:
\begin{quote}
(1) $f(1)=1$ \newline
(2) $f(x+1)=xf(x)$ \newline
(3) Funktio $f(x)$ on logaritmisesti konveksi funktio.
\end{quote}

Todistus
\newline
Ehdot (1) ja (2) on jo todistettu, joten jäljelle jää todistaa ehto (3) todeksi. 

\newpage
\section{Sovelluskohteita}
\newpage
\section{Kirjallisuutta}
\onehalfspacing
[1]


\end{document}