\documentclass[12pt]{article}
\usepackage[utf8]{inputenc}
\usepackage[finnish]{babel}
\usepackage[T1]{fontenc}
\usepackage[utf8]{inputenc}
\usepackage{setspace}
\usepackage{amsmath}
\usepackage{amssymb}
\usepackage{amsthm}
\theoremstyle{definition}
\newtheorem{maar}{Määritelmä}
\theoremstyle{plain}
\newtheorem{lause}[maar]{Lause}
\usepackage{csquotes}
\numberwithin{equation}{section}
\setlength{\parskip}{\medskipamount}
\setlength{\parindent}{0pt}
\setlength{\emergencystretch}{15pt}
\thispagestyle{empty}                   

\newcommand{\ee}[1]{e^{#1}}             % makromäärittely;
                                        % hakasuluissa parametrien lukumäärä

\newcommand{\viiva}{\mathop{\Bigg/}}
\newcommand{\sij}[3]{\viiva\limits_{\hspace*{-5mm}{#1}}^{\hspace*{5mm}{#2}}{#3}}

\title{Gamma funktio}
\author{Reetta Tervo}
\begin{document}
\maketitle

\newpage
%\setcounter{tocdepth}{1}
\tableofcontents

\newpage
\section{Johdanto}
\onehalfspacing
Tämän tutkielman vatoitteena on esitellä gammafunktio. Keskeisenä osana gammafunktiota on Bohr-Mollerupin lause, joka todistetaan...

\newpage





\section{Gammafunktio}
Yleisin esitysmuoto gammafunktiolle on Leonard Eulerin vuonna 1729 määrittämä integraaliesitys
\begin{equation}\label{yhtalo:gammafunktio}
     \Gamma(x) = \int_{0}^{\infty} e^{-t} t^{x-1} dt.
\end{equation}

Gammafunktio on epäoleellinen integraali. Tästä johtuen, tutkittaessa sen olemassa oloa, tulee sen olla määritelty hyvin. Tällöin tutkitaan siis määrättyjä integraaleja
\begin{align*}
    (a) \int_{\epsilon}^{1} e^{-t} t^{x-1} dt \\
    (b) \int_{1}^{\delta} e^{-t} t^{x-1} dt.
\end{align*}

Tutkitaan ensin määrättyä integraalia $(a)$, kun $x>0$:\newline
Kun luku $t$ on positiivinen, niin integtoitava osuus on pienempi kuin $t^{x-1}$. Näin saadaan
\begin{align*}
    e^{-t} t^{x-1} < t^{x-1}.
\end{align*} 
Tällöin voidaan arvioida integraalia ylöspäin seuraavasti 
\begin{align*}
    \int_{\epsilon}^{1} e^{-t} t^{x-1} dt < \int_{\epsilon}^{1} t^{x-1} d = \Big/_\epsilon^1 {t^x}x^{-1} = \frac{1}{x}-\frac{\epsilon^{x}}{x}.
\end{align*}
Kun $\epsilon \rightarrow 0$, niin $\epsilon^{x}x^{-1}\rightarrow 0$ ja tällöin myös $x^{-1}-\epsilon{^x}x^{-1} \rightarrow \frac{1}{x}$. Siispä, kun $\epsilon\rightarrow0$, niin epäoleellinen integraali on  rajoitettu välillä $(0, 1]$ ja raja-arvo
\begin{align*}
    \lim_{\epsilon\to\0}\int_\epsilon^{1}e^{-t}t^{x-1}dt
\end{align*}
on olemassa. Täten voidaan todeta, että epäoleellinen integraali $\int_{0}^{1} e^{-t} t^{x-1} dt$ on olemassa ja se suppenee. \newline

Tutkitaan seuraavaksi määrättyä integraalia $(b)$, kun $x>0$: \newline
Kun luku $t$ on positiivinen, niin kaikki sarjan $e^t$ termit ovat positiivisia. Tällöin myös kaikilla kokonaisluvuilla $n$ pitää paikkaansa, että $e^t\ge\frac{t^n}{n!}$. Täten päästään epäyhtälöön $e^{-t}\le\frac{n!}{t^n}$ ja tästä edelleen $e^{-t}t^{x-1}\le{t^{-{n+1-x}}}{n!}$. Valitaan jonkin $n>x+1$, jolloin saadaan epäoleelliselle integraalille yläraja $\delta=\frac{n!}{n-x}$,
\begin{align*}
    \int_1^{\delta}e^{-t}t^{x-1}dt.
\end{align*}
Koska integraalin arvo kasvaa, kun $\delta$ kasvaa eli se kasvaa monotonisesti, niin raja-arvo 
\begin{align*}
    \lim_{1\to\infty}\int_1^{\delta}e^{-t}t^{x-1}dt
\end{align*}
on olemassa. Tällöin myös epäoleellinen integraali $\int_1^{\infty}e^{-t}t^{x-1}dt$ on olemassa.\newline

Koska määrätyt integraalit $(a)$ ja $(b)$ ovat olemassa, niin tällöin myös epäoleellinen integraali $(2.1)$ on olemassa. Siis gammafunktio on olemassa.



\subsection{Kertomafunktio}
\begin{maar}
(kertomafunktio)
Funktio $f: N \rightarrow N$ on kertomafunktio, jos seuraavat ehdot toteutuvat:
\begin{quote}
    i) $f(1)=1$ \newline
    ii) $f(n+1)=(n+1)f(n)$ \newline
    Kun ehdot $i$ ja $ii$ toteutuvat, niin voidaan merkitä $f(n)=n!$.
\end{quote}
\end{maar}
Kertomafunktio on monelle tuttu jo entuudestaan. Sille ominaista on arvojen erittäin nopea kasvu. Esimerkiksi: $4! = 1\cdot 2\cdot3\cdot4=24$, kun taas $5! = 1\cdot2\cdot3\cdot4\cdot5=120$.
Määritelmän 1, kohta $i)$ on selkeä, sillä luvun $1$ kertoma on luku itse eli $1!=1$. Kohta $ii)$ puolestaan on mielenkiintoinen, koska sen mukaan kertomafunktiolla on ominaisuus, että 
\begin{equation}
    (n+1)! = (n+1)n!.
\end{equation}
Koska gammafunktio on kertomafunktion yleistys, niin sille pätevät täysin samat ehdot kuin kertomafunktiolle. Gammafunktio siis toteuttaa kertomafunktion eli määritelmän 1 ehdot. Yhtälöstä $\eqref{yhtalo:gammafunktio}$ saadaan tutkittua ensimmäinen ehto laskemalla integraali arvolla $1$. Tällöin
\begin{equation*}
    \Gamma(1)=\int_0^\infty e^{-t}t^{1-1}dt=\int_0^\infty e^{-t}t^{0}dt = \int_0^\infty e^{-t}dt = 1.
\end{equation*}

\newpage





\section{Bohr-Mollerupin lause}
\onehalfspacing
Bohr-Mollerupin lauseen mukaan gammafunktio on ainoa kertomafunktion yleistys, joka on logaritmisesti konveksi. Tämä ominaisuus tekee gammafunktiosta erityisen. Funktio on konveksi, kun sen kuvaaja on alaspäin kupera. Bohr-Mollerupin lauseen mukaan gammafunktion logaritmi on siis kuvaaja, joka on alaspäin kupera.
\newline

\begin{maar}
(konveksisuus) \label{maar: konveksisuus}
Olkoon $X\subset\mathbb{R}, t \in (0, 1)$ ja funktio $f: X \rightarrow \mathbb{R}$. Funktio $f$ on \emph{konveksi}, jos kaikilla $x,y \in X$
\begin{equation}\label{yhtalo:konveksi}
    f(tx+(1-t)y) \le tf(x)+(1-t)f(y).
\end{equation}
\end{maar}


\begin{maar}
(logaritminen konveksisuus) \label{maar: logkonveksisuus}
Olkoon $X\subset\mathbb{R}, t \in (0, 1)$. Funktio $f: X \rightarrow \mathbb{R}$ on siis määritelty ja positiivinen. Funktio $f$ on \emph{logaritmisesti konveksi}, jos $logf(x)$ on \emph{konveksi}.
\end{maar}

\begin{lause} \label{lause: young}
(Youngin epäyhtälö). Oletetaan, että $a$ ja $b$ ovat positiivisia reaalilukuja ja $q \in (0,1)$. Tällöin
\begin{equation}
    a^{q}b^{1-q} \le qa+(1-q)b
\end{equation}
\end{lause}
\begin{proof}
Olkoon $a,b \ge 0$ ja $q \in (1,0)$. Kun luku jompi kumpi luvuista $a$ tai $b$ on nolla, niin epäyhtälö on tosi. Yhtäsuuruus pätee, jos ja vain jos $a=b$. Oletetaan, että $a,b > 0$. Käytetään apuna logaritmia ja tietoa siitä, että logaritmifunktio on konveksi.Tällöin saadaan
\begin{equation*}
    \ln(a^{q}b^{1-q}) = \ln(a^q) + \ln(b^{1-q}) = q \ln a+(1-q)\ln b \le \ln(qa+(1-q)b).
\end{equation*}
Yllä sovellettiin konveksisuuden määritelmän epäyhtälöä $\eqref{yhtalo:konveksi}$. Koska tiedetään, että logaritmifunktio on myös aidosti kasvava, niin saadaan 
\begin{equation*}
    a^{q}b^{1-q}\le qa+(1-q)b.
\end{equation*}
\end{proof}

\begin{lause} \label{lause: hölder}
(Hölderin epäyhtälö) Oletetaan, että $t \in (0,1)$. Tällöin 
\begin{equation}
    \int_{\mathbb{R}} |f(x)g(x)|dx \le \left( \int_{\mathbb{R}} |f(x)|^{\frac{1}{t}} \right)^{t}\left( \int_{\mathbb{R}} |g(x)|^{\frac{1}{1-t}}\right)^{1-t}.
\end{equation}
\end{lause}
\begin{proof}
Kun $\int_\mathbb{R}|f(x)|^{\frac{1}{t}}dx$, niin on oltava $f(x)=0$. Tällöin pätee myös, että
\begin{equation}
    \int_{\mathbb{R}} |f(x)g(x)|dx=0.
\end{equation}
Vastaavasti myös $g(x)=0$, kun $\int_{\mathbb{R}}|g(x)|^{\frac{1}{1-t}}dx=0$. \newline
Olkoon $t\in(0,1).$ Käytetään nyt \emph{Youngin epäyhtölöä}, \eqref{lause: young} lauseen todistuksessa, jolloin saadaan
\begin{equation}
    \frac{|f(x)||g(x)|}{(\int_{\mathbb{R}}|f(x)|}
\end{equation}

\end{proof}

\begin{lause}
(gammafunktio) Gammafunktio  toteuttaa seuraavat ehdot:
\begin{quote}
    i) $\Gamma(1)=1$ \newline
    ii) $\Gamma(x+1)=x\Gamma(x)$ \newline
    iii) $\Gamma(x)$ on logaritmisesti konveksi.
\end{quote}
\end{lause}

\begin{proof}
Ehto $i)$ saadaan suoraan sijoittamalla luku $1$ yhtälöön \eqref{yhtalo:gammafunktio}, kun $x>0$:
\begin{align*}
    \Gamma(1) = \int_0^\infty e^{-t}t^{1-1} dt = \int_0^\infty e^{-t}dt = 1.
\end{align*}
Ehto $ii)$ puolestaan voidaan osoittaa todeksi, kun sijoitetaan yhtälöön \eqref{yhtalo:gammafunktio} $x$:n paikalle $x+1$, kun $x>0$.
\begin{equation}\label{yhtalo:joku}
    \Gamma(x+1)=e^{-t}t^{(x+1)-1} = e^{-t}t^{x}
\end{equation}
Nyt osittaisintegroidaan integraali \eqref{yhtalo:joku}:
\begin{align*}
    \Gamma(x+1) & = \int_{\epsilon}^{\delta} e^{-t} t^{x} dt \\
    & = \Big/_\epsilon^\delta -e^{-t}t^{x}+x\int_\epsilon^\delta e^{-t}t^{x-1}dt \\
    & = e^{-\epsilon}\epsilon^{x}-e^{-\delta}\delta^{x}+x\int_\epsilon^\delta e^{-t}t^{x-1}dt.
\end{align*}

Kun $\epsilon\rightarrow0$, niin $\epsilon^x\rightarrow0$ ja tällöin myös $e^{-\epsilon}\epsilon^x \rightarrow0$. Tällöin osittaisintegraalin tuloksen ensimmäinen termi katoaa. Tämän lisäksi tiedetään, että $e^{-\delta}\delta^{x} < \frac{n!}{\delta^{n-x}}$. Nyt, kun $\delta\rightarrow\infty$, niin luku $\frac{n!}{\delta^{n-x}}\rightarrow0.$ Tällöin myös $e^{-\delta}\delta{^x}\rightarrow0$ ja toinenkin termi katoaa. Kun kaksi ensimmäistä termiä katoavat, niin jäljelle jää vain 
\begin{align*}
   \Gamma(x+1)=x\int_\epsilon^\delta e^{-t}t^{x-1}dt.
\end{align*}

Siis $\Gamma(x+1) = x\Gamma(x)$ eli ehto $ii)$ on myös tosi.\newline

Kolmas todistus on huomattavasti haastavampi. Ehdon $iii)$ mukaan gammafunktion, $\Gamma(x)$, tulee olla logaritmisesti konveksi. Funktion $\Gamma(x)$ tulee siis olla määritelmän 3 mukainen. Käytännössä tämä tarkoittaa sitä, että funktion $log\Gamma(x)$ tulee olla konveksi.\newline

Merkitään $g(x) = log\Gamma(x)$. Funktion $g(x)$ tulee olla \emph{koveksi}, jotta ehto $iii)$ toteutuu.
\end{proof}

\begin{lause}
(Bohr-Mollerupin lause)
\newline
Jos funktio $f(x)$ toteuttaa seuraavat kolme ehtoa, niin se on gammafunktio:
\begin{quote}
(1) $f(1)=1$ \newline
(2) $f(x+1)=xf(x)$ \newline
(3) Funktio $f(x)$ on logaritmisesti konveksi funktio.
\end{quote}
\end{lause}

\begin{proof}
Ehdot (1) ja (2) on jo todistettu, joten jäljelle jää osoittaa ehto (3) todeksi. 
\end{proof}
\newline

\newpage
\section{Kirjallisuutta}
\onehalfspacing
[1] 
\newline
[2]
\newline
[3]
\newline
[4]


\end{document}